\chapter{Modello Matematico}\label{cap:modello}
Il modello matematico riguarda tutta la parte generativa dei valori delle risorse e dell'indice di occupazione dei nodi. Molte delle relazioni che sono state utilizzate sono state ricavate empiricamente grazie ad un operazione di \var{tuning} che ci ha consentito di raggiungere una situazione il più reale possibile.
\section{Valori risorse}
Le risorse del sistema realizzato possono essere suddivise in due gruppi: dipendenti dalla presenza dello SC (e Dsm) e indipendenti dalla loro presenza. Quanto detto è stato fatto per rappresentare più realmente un caso normale. Si consideri il seguente esempio, ovvero un nodo mobile con connessione \var{wireless} e un nodo fisso con connessione \var{wired}. Il nodo mobile viene collegato e scollegato dalla rete elettrica quando necessario. Sul nodo mobile sono in esecuzione delle normali operazioni di richiesta di informazioni verso il nodo fisso. Sul nodo fisso è in esecuzione il server che risponde alle risposte di vari nodi mobili. Lo SC è al momento sul nodo fisso, questo implica un carico computazione molto elevato per tale nodo, quindi se necessario tale agente SC migrerà sul nodo mobile che al momento ha molte risorse disponibili. Per consentire questo meccanismo di migrazione è stato necessario realizzare delle formule di generazione adattabili al loro stato attuale (presenza o non presenza dello SC oppure collegato o scollegato dalla rete elettrica). Tutti i valori generati rappresentano delle percentuali. Di seguito vengono riportate tutte le formule usate per la generazione.
\subsection{Cpu}
La risorsa \var{cpu} dipende fortemente dalla presenza dello SC, infatti si è deciso di applicare la seguenti relazioni, la prima nel caso ``senza SC''
\begin{equation}
cpu = (\alpha \cdot 100) mod 60
\end{equation}
in cui $\alpha$ rappresenta un valore casuale da 0 a 1. Il modulo 60 è stato usato per limitare il valore a 60. Nel caso ``con SC'' invece è stata usata la seguente relazione:
\begin{equation}
cpu = ((\alpha \cdot 100) mod 60) + 40
\end{equation}
in cui si aggiunge il valore 40 che rappresenta il carico supplementare dovuto alla presenza dello SC.
\subsection{Ram}
La risorsa viene generata come per la cpu.
\begin{equation}
ram = (\alpha \cdot 100) mod 60
\end{equation}
Nel caso ``con SC'' invece è stata usata la seguente relazione:
\begin{equation}
ram = ((\alpha \cdot 100) mod 60) + 40
\end{equation}
\subsection{Memory}
La memoria ha un comportamento leggermente più complesso, infatti è stato realizzato un meccanismo di riempimento e svuotamento della memoria dipendente dalla presenza dello SC. All'avvio del nodo viene generato un valore iniziale tramite la seguente formula:
\begin{equation}\label{eq:memory}
memory = (\alpha \cdot 100) mod 60
\end{equation}
Ad ogni \var{tick} si modifica il valore nel seguente modo:
\begin{equation}
memory = memory + 0.2 \cdot ((\alpha \cdot 100) mod 10)
\end{equation}
Quest'ultima relazione consiste nell'aggiungere al valore precedente il $20\%$ di un valore modulo 10, grazie a ciò si ottiene un incremento molto lieve dell'utilizzazione della memoria.
Nel caso ``senza SC'' invece si ha
\begin{equation}
memory = memory - 0.2 \cdot ( ( \alpha \cdot 100 ) mod 10 )
\end{equation}
a differenza del caso precedente si sottrae il nuovo valore da quello vecchio, questo sta ad indicare una diminuzione dell'utilizzazione. Inoltre nel caso in cui la memoria raggiunge valori minori di 0 o maggiori 100 si rigenera il valore della memoria con la formula usata inizialmente (ved. \ref{eq:memory}).
\subsection{Energy}
Nel caso del componente riguardante l'energia sono state applicate delle formule più adattative, in quanto il suo valore dipende dal collegamento o scollegamento alla rete elettrica e dalla presenza o non presenza dello SC. Inoltre dipende dal tipo di connessione, ovverro se è wireless o wired. All'avvio viene generato il valore tramite la seguente formula.
\begin{equation}\label{eq:energy}
memory = (\alpha \cdot 100) mod 60
\end{equation}
Ad ogni tick dell'agente si può avere una delle seguenti formule in base alle condizioni. In caso di connessione wired si ha:
\begin{equation}
energy = 100
\end{equation}
considerando che il wired è stato assunto come nodo fisso collegato alla rete elettrica. In caso di connessione wireless si deve distinguere dal caso in cui è connesso alla rete elettrica (\code{powerOn}) e il caso in cui non lo è (\code{!powerOn}). Per rendere reale i valori generati si è realizzato un meccanismo per gestire la connessione e sconnessione dalla rete elettrica, ovvero:
\begin{gather}
\text{se }energy \leq 10\text{ then }powerOn = true \\
\text{se }energy \geq 99\text{ then }powerOn = false
\end{gather}
Con la rete elettrica collegata si ha:
\begin{gather}
energy = energy + 0.8 \text{ se SC non presente},\\
energy = energy + 0.6 \text{ se SC presente}
\end{gather}
in caso di rete elettrica scollegata si ha:
\begin{gather}
energy = energy - 0.2 \text{ se SC non presente},\\
energy = energy - 1 \text{ se SC presente}
\end{gather}
\subsection{Latency}
Nel caso della latenza è stato usato un meccanismo dipendente solo dalla presenza dello SC e dal tipo di connessione. In particolare se nel nodo è presente lo SC oppure ha una connessione wireless si ha:
\begin{equation}
latency = ((\alpha_{1} \cdot 100)mod60) + \left( \left(\frac{\alpha_{2} \cdot 100}{\frac{\beta}{50}}\right)mod40\right)
\end{equation}
come si può notare vengono generati due numeri, uno modulo 60 ed uno modulo 40 in modo da avere un numero massimo di 100. Inoltre è presente un valore $\beta$ che rappresenta la banda del nodo.
\subsection{Reliability}
Il valore della affidabilità del canale viene generato nello stesso modo della latenza, infatti si ha:
\begin{equation}
reliability = ((\alpha_{1} \cdot 100)mod60) + \left( \left(\frac{\alpha_{2} \cdot 100}{\frac{\beta}{50}}\right)mod40\right)
\end{equation}
\subsection{ReqInterval}
In quest'ultima risorsa la generazione avviene molto semplicemente generando un numero casuale da 0 a 100.
\begin{equation}
reqInterval = \alpha \cdot 100
\end{equation}
\section{SLA Checker}
Lo SC essendo il componente centrale di tutto il sistema ha richiesto un lavoro molto accurato per la realizzazione di un buon algoritmo e di un indice di valutazione efficiente. Per algoritmo si intendono le operazioni che vengono svolte dallo SC ciclicamente (ad ogni tick dell'agente), ovvero il controllo sulla validità dei contratti e il controllo della necessità di migrazione. Per indice di valutazione si intende il valore che viene generato per ogni nodo dai valori del suo contesto, tale indice viene utilizzato dallo SC per trovare il nodo su cui eventualmente migrare.
\subsection{Algoritmo controllo}
L'algoritmo realizzato consente il controllo della violazione dei contratti e il controllo della necessità di migrazione. In particolare tali operazioni possono essere effettuate grazie alla conoscenza di tutte le informazioni sui nodi del sistema, ovvero il contesto e i contratti. L'algoritmo è suddiviso in due parti, la prima si occupa del controllo della validità dei contratti.
\subsubsection{SLA Contract checking}
\begin{enumerate}
	\item si prende un contratto SLA dalla lista dei contratti;
	\item si recuperano i dati del contesto dei due partecipanti al contratto (Publisher, Subscriber);
	\item si controlla la validità del contratto tramite la relazione seguente:
		$$
			latency_{pub} > latency_{sla}
		$$
			\centering{OR}
		$$
			reliability_{pub} > reliability_{sla}
		$$
			\centering{OR}
		$$
			reqInt_{sub} > reqInt_{sla}
		$$
	\item se tale relazione risulta valida allora il contratto è stato violato.
\end{enumerate}
Dalla relazione riportata si può notare la presenza dei valori di \code{latency}, \code{reliability} e \code{reqInt} che hanno come pedice \code{pub}, \code{sub} e \code{sla} che indicano rispettivamente il Publisher, il Subscriber e il contratto SLA. La relazione riportata non fa altro che controllare se il publisher viola il contratto non garantendo la latenza e l'affidabilità precedente, oppure se il Subscriber viola il contratto effettuando più richieste di quante stabilite dal contratto SLA.
\subsubsection{SC Migration checker}
La seconda parte di cui si occupa lo SC è la verifica della necessità di migrazione su altri nodi. In particolare tale operazione consiste nel valutare l'indice del contesto di ogni nodo, controllando se il valore supera una certa soglia prefissata, nel qual caso sarebbe necessario migrare su un nuovo nodo. Il calcolo dell'indice di utilizzazione viene effettuato con la seguente formula:
\begin{equation}\label{eq:index}
index = cpu \cdot 0.23 + ram \cdot 0.23 + memory \cdot 0.23 + (100-energy) \cdot 0.3
\end{equation}
Il valore $index$ calcolato rappresenta una percentuale che va da 0 a 100 che indica l'utilizzazione del nodo valutato. L'indice è il fulcro dell'algoritmo di \var{Migration checking}, in quanto la migrazione si basa proprio sul confronto di tale valore. Verranno ora riportati i passi dell'algoritmo:
\begin{enumerate}
	\item Si recupera il nodo migliore su cui migrare tramite la funzione \code{getBestNode()};
	\item si seleziona il contesto attuale del nodo in cui si trova lo SC;
	\item si verifica che l'indice del contesto attuale sia maggiore dell'indice di migrazione;
	\item se la condizione è verificata si effettua la migrazione sul nodo risultato migliore;
	\item si notifica, tramite dsm, l'avvenimento della migrazione a tutti nodi presenti;
	\item se la condizione non si verifica non viene effettuata la migrazione.
\end{enumerate}
\subsubsection{Best Node}
La funzione \code{getBestNode()} riportata nell'algoritmo precedente viene descritta dettagliatamente di seguito:
\begin{lstlisting}[frame=trBL]
AID next, iter;
Enumeration<AID> keys = sc.getContextTable().keys();
Context context;
float cpu, ram, memory, energy;
float total=500, iter=0;

while(keys.hasMoreElements()) {
  next = keys.nextElement();
  context = sc.getContextTable().get(next);
  if(context!=null) {
    cpu = context.getCpu();
    ram = context.getRam();
    memory = context.getMemory();
    energy = context.getEnergy();
    iter = cpu + ram + memory + (1-energy);
    if(cpu < Context.CPU_LIMIT && ram < Context.RAM_LIMIT && memory < Context.MEMORY_LIMIT && energy > Context.ENERGY_LIMIT && iter < total) {
      best = next;
      total = iter;
    }
  }
}
\end{lstlisting}
Lo scopo dell'algoritmo riportato è di recuperare il contesto che ha i valori ``migliori'', ovvero che si attiene alla seguente relazione:
$$
cpu < cpu_{limit} \;AND\; ram < ram_{limit}\; AND
$$
$$
memory < memory_{limit} \;AND\; energy > energy_{limit}\; AND\; \theta < min
$$
in cui i valori con pedice $limit$ indicano il massimo numero per cui si accetta il valore come ``migliore''. $\theta$ invece rappresenta il minimo attuale, calcolato tramite \ref{eq:iter}
\begin{equation}\label{eq:iter}
\theta = cpu + ram + memory + (1-energy)
\end{equation}
in ogni ciclo viene calcolato tale valore e se risulta essere minore del minimo attuale si sostituisce $min$ con tale valore.
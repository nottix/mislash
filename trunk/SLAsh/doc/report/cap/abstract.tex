\chapter*{Abstract}
\addcontentsline{toc}{chapter}{Abstract}
Nel mondo dell'Informatica si è diffuso da molto tempo il bisogno di sviluppare applicazioni per dispositivi mobili, quali notebook, smartphone, tablet ed altro ancora; in particolare si è raggiunta la necessità di sviluppare applicazioni distribuite per dispositivi mobili, ovvero il \var{Mobile Computing}. Grazie a questo bisogno sono nate molte nuove tecniche che hanno apportato migliorie alla comunicazione tra dispositivi, come la stipulazione di \var{SLA} (Service Level Agreement: Contratti basati sul livello di servizio) tra richiedenti e fornitori di servizi, ed infine le \var{dsm} (Distribuited Shared Memory). Durante il corso di Informatica Mobile è stato richiesto di realizzare un sistema che simulasse il comportamento di alcuni richiedenti e fornitori dopo aver stipulato un contratto sulla qualità del servizio, sfruttando \var{dsm} per la comunicazione dei dati. In questa relazione verranno trattate le problematiche incontrate nella progettazione del sistema, ed in particolare tutte le scelte implementative effettuate, nonchè i modelli matematici ed empirici. Inoltre verranno presentati anche dei test di esecuzione con i relativi dati sulle performance; infine si riporteranno le conclusioni a cui si è giunti ed i possibili sviluppi futuri.
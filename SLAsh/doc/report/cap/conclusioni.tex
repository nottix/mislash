\chapter{Conclusioni e sviluppi futuri}\label{conclusioni}
Il sistema realizzato ha consentito di valutare la presenza di un container dsm in un ambiente mobile con contratti SLA. Nello specifico si è valutato con quale frequenza lo SC e il DSM migrino tra differenti nodi. Sono stati eseguiti più test e si è variato i parametri per raggiungere uno scenario il più reale possibile, cercando un compromesso tra efficienza e qualità. (ROSCIO MERDA) Dalle simulazioni effettuati si è giunti alla conclusione che una migrazione tra i vari nodi consente un aumento dell'efficienza di tutti i nodi, in quanto ognuno partecipa alla computazione equamente dando il proprio contributo. Il quale contributo è proporzionale al tipo di nodo, ovvero dipende dal tipo di connessione, dalla banda e dalla presenza della rete elettrica. Infine si è pensato che in futuro sarebbero possibili ulteriori sviluppi come l'aggiunta di altre politiche di migrazione, anche se l'attuale ha dato ottimi risultati. Inoltre un'altra miglioria potrebbe essere apportata al DSM, cercando di evitare la serializzazione completa su database, il che potrebbe essere fatto serializzando solamente le scritture.
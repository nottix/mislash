\chapter{Introduzione}\label{cap:introduzione}
Per comprendere a fondo l'entità di tale progetto deve essere riportata la specifica del problema da risolvere, in quanto consente di entrare nell'ottica del problema.
\section{Specifiche problema}
\subsection{Logica applicazione}
Si richiede di progettare una architettura di supporto al monitoraggio e controllo di SLA\cite{gino} in ambiente (possibilmente) mobile. L’architettura del servizio è basata sulla definizione di un certo numero di componenti “logici”: SLA Checker (SC), Context Manager (CM), Resource Monitor (RM). Tali entità interagiscono tra loro unicamente tramite il meccanismo DSM ("tuple space"). Il ruolo di tali entità viene descritto come segue:
\begin{description}
\item[SLAchecker (SC):] data una coppia fornitore/richiedente servizio, che ha stipulato un SLA, SC ha il compito di controllare il rispetto dei parametri del contratto sia da parte del fornitore che del richiedente, e segnalare eventuali violazioni ad entrambi. A questo scopo, SC raccoglie informazioni fornite da opportuni componenti di tipo Monitor presenti sia sul nodo del fornitore che del richiedente, relative a (per esempio):
\begin{itemize}
\item tempo di risposta osservato per una richiesta;
\item affidabilità (completamento con successo) di una richiesta;
\item intervallo di tempo tra due richieste consecutive.
\end{itemize}
I dati “grezzi” ricevuti dai componenti di monitoraggio vengono elaborati da SC per calcolare i valori degli indici di interesse.
\item[Context Manager (CM):] è un componente associato a un particolare nodo di elaborazione e il suo ruolo è quello di fornire informazioni su vari tipi parametri che caratterizzano il contesto di esecuzione di componenti presenti su quel nodo, p.es.:
\begin{itemize}
\item utilizzazione cpu;
\item RAM disponibile;
\item memoria stabile (disco, o altro) disponibile;
\item tipo di rete e banda disponibile;
\item energia disponibile.
\end{itemize}
\item[Resource Monitor (RM):] un componente di questo tipo fornisce le informazioni relative a una delle risorse elencate sopra.
\end{description}
\subsection{Ambiente d'uso}
L’ambiente in cui si immagina che il servizio di controllo di SLA venga realizzato è costituito, in generale, da una molteplicità di nodi (fissi o mobili) con vari livelli di disponibilità di risorse interne (memoria, cpu, ecc.), connessi tra loro da infrastrutture di comunicazione di varia qualità. Su tali nodi sono in esecuzione componenti che offrono/richiedono servizi. Ogni volta che una coppia fornitore/richiedente stipula un SLA, il controllo di questo SLA viene affidato a un componente SC.
\subsection{Lavoro progettuale}
Si richiede di progettare e realizzare, utilizzando la piattaforma JADE (\url{http://jade.tilab.com}), l’architettura indicata nella sezione precedente. In particolare, occorre definire una localizzazione dei componenti e organizzazione del modello DSM (basato sulla realizzazione di uno o più "tuple space") che sia adeguata alla esecuzione del servizio di controllo SLA in un ambiente possibilmente mobile, caratterizzato da possibile scarsità di risorse per i componenti in esecuzione su determinati nodi. Il livello di adeguatezza andrà valutato rispetto alla capacità di ottimizzare misure di prestazione quali:
\begin{itemize}
\item traffico generato su rete;
\item consumo di energia da parte di nodi mobili;
\item carico computazionale/di memorizzazione per nodi mobili;
\end{itemize}
tenendo anche conto del fatto che il contesto (disponibilità di risorse) in cui opera il servizio di controllo SLA può variare nel tempo, per esempio per effetto della mobilità di alcuni nodi.
